\documentclass[a4paper,11pt]{article}
\usepackage[osf]{mathpazo}
\usepackage{ms}
\usepackage[]{natbib}
\raggedright

\newcommand{\datastorr}{\texttt{datastorr}}
\newcommand{\smurl}[1]{{\footnotesize\url{#1}}}
\newcommand{\ghsmurl}[1]{{\footnotesize\href{https://github.com/#1}{#1}}}

\usepackage{graphicx}

\title{Leveraging open platforms to manage and distribute streams of versioned data}
\author{Richard G. FitzJohn$^1$, Daniel S. Falster$^2$,\\ Matthew
  W. Pennell$^3$, and William K. Cornwell$^{2,*}$}
\affiliation{
$^1$ Imperial College, London, United Kingdom\\
$^2$ Evolution \& Ecology Research Centre, School of Biological, Earth and Environmental Sciences,\\
University of New South Wales, Sydney, NSW 2052, Australia\\
$^3$ Department of Zoology and Biodiversity Research Centre,\\
University of British Columbia, Vancouver, B.C. V6T 1Z4 Canada\\
$^*$ Corresponding author: w.cornwell@unsw.edu.au\\
}
\date{}

\bibliographystyle{mee}

\usepackage[title,titletoc,toc]{appendix}

\mstype{Article Note}
\runninghead{Lightweight versioned data}
\keywords{}

\begin{document}
\mstitlepage
\noindent
\parindent=1.5em
\addtolength{\parskip}{.3em}
\doublespacing
\linenumbers
\section{Summary}
\begin{enumerate}
\def\labelenumi{\arabic{enumi}.}
\itemsep1pt\parskip0pt\parsep0pt
\item Data is cool.
\item Recognise need for versioned datasets
\item Provide workflow / infrastructure for people to distribute and access small-medium sized data with minimal setup and cost, by leveraging emerging technologies
\end{enumerate}

\subsection{Introduction}

Data is now a first-class scientific product. Funding bodies, publishers, and scientific social norms are increasingly both requiring data to be published alongside research publications and recognizing the value of producing and sharing data without any accompanying analyses. This has been a rapid, and exciting, sea change and the scientific community is still grappling with how best to disseminate data and credit data creators and aggregators [Cite some Open Science papers, both pro and con]. One thing that is clear is that there is no single best solution for managing and distributing data, as data types and use is incredibly heterogeneous. On one end of the spectrum are massive, centralized repositories that are constantly being added to; this is a great solution to storing and distributing relatively standardized data such as sequences (e.g., \texttt{GenBank}, \texttt{SRA}) (Table \ref{tab:sql_v_versioneddata}). On the other end are repositories for storing a wide variety of static, stand-alone datasets (\texttt{Dryad}, \texttt{Figshare}, and less optimally, as supplements to research papers)(Table \ref{tab:sql_v_versioneddata}). We suggest there is a substantial middle ground that is not well served by either of these existing paradigms, where a lightweight system for serving versioned datasets is needed.

In this article we i) introduce the concept of Lightweight Versioned Data (LVD) and describe why it is needed; ii) outline how emerging technologies in data science (Table \ref{tab:technologies}) can be leveraged to help researchers maintain, distribute, and access small to medium sized datasets; and iii) introduce a new R package called \texttt{datastorr}, an implementation of a LVD system. While we emphasize particular technologies in our implementation, the principles are general and could easily be ported to other platforms. Furthermore, our backgrounds in the life sciences (and ecology and evolution, in particular); however, we argue that the issues we identify and the solutions we put forward are likely to be relevant to a wide variety of domains, across the natural and social sciences.

\subsection{Why versioning of data is needed}

Almost all useful databases are living--they \emph{should} get better
through time either by the addition of new data or by improving the
quality of existing data. For example a dataset on plant characteristics
might become outdated as taxonomic changes occur. Updating the dataset might only
require a small amount of work, but if there is no
way to distribute the updated dataset, then that work has to be re-done by
every user of that dataset. Moreover, those required changes add steps
between the canonical (or master) dataset and the new analysis (and publication),
making reproducibility of the new analysis more difficult than it needs
to be. If the canonical dataset is versioned--and versions are
accessible--the up-to-date data can be distributed, used, and analyses
based on any of the released versions data can be reproduced.

\subsection{Why versioning of data has been difficult}

While the benefits of versioned data are fairly clear, determining how best to implement it has been anything but. The static model of data distribution widely used in science, where datasets are archived either as supplementary materials for a journal article
or under its own DOI, is simply not designed to provide streams of versioned
data (Table \ref{tab:sql_v_versioneddata}). A particular issue with the static
model is that data releases are mostly triggered by publication of academic
papers. But for a living dataset, there may be multiple releases made in
between papers. This is particularly the case with the discovery of small
errors. For any medium-sized dataset that is put to reuse, errors will
inevitably be found, and there needs to be some easy  way to correct them and
distribute the updated dataset.

To overcome the limitations of static datasets, many research groups have
setup dedicated webservers to host and distribute data. In these systems,
datasets areas maintained in a structure database, such as SQL, and a front
end to this database is written, often using a platform such as Ruby on rails.
When designed well, such systems are very useful for updating or adding to a
dataset. However, the infrastructure investment (designing maintaining, and
hosting the database) is clearly overkill for most projects and is a very
substantial barrier to entry. Furthermore, the system requires ongoing funds to maintain it, making it often unsustainable for many groups. Moreover, the existence of a live webserver,
does not in itself guarantee ongoing access to different versions of a
dataset.

When considering the right tool to store and deliver versioned data, the size
of the dataset is another essential consideration. ``Big data'' brings with it
specific needs and so requires a specific set of hardware and software tools.
Examples of big data include genetic datasets, global datasets of species
observations, and remote sensing data. This size of this data is bigger than
that which individual research groups can maintain and often requires a
specialized governmental or non-governmental institution with full-time staff
to curate and maintain those research resources. We were particularly
interested in smaller and more specialized problems--the type of datasets that
a given research group might specialize in. In most of these cases, the data
collected will not be ``big'' but rather small to medium sized. Such data
support research projects, but are not in most cases general enough to
establish infra-structure for. But crucially the data themselves will change
over time. And currently there is no infrastructure for organizing and
distributing these types of ``living datasets''.

So what is needed is simple tool-chain for
making versions of a dataset and delivering it to potential users, along
with notes on what has changed and why since the previous version.
Table \ref{tab:user_requirements} outlines what we consider requirements for
three groups of people interacting with this system: maintainers,
contributors, users. The solution is ideally low-cost (both in money and
time), sophisticated, yet requiring little skill to setup and use.  Ideally
there would also be a way for users to also become contributors, by flagging
errors for the curator to correct in the next release, or even for the user to
fix themselves.

% For small datasets, it is simply not possible to use the same tools developed for
% big data: there are a number of high-setup-cost options (stand-alone
% websites, APIs), which have advantages if you have the resources to
% develop and maintain them. We wanted a relatively low cost way (both in
% money and time).

\section{A lightweight, cheap and scalable workflow for versioned data}

With the growth of open-source tools and collaborative platforms, we believe
all the features needed for delivering the vision outlined above are now
present. The workflow we describe below builds off existing platforms and
combines them in new ways, and offers the potential for
widespread uptake at little or no cost.

In brief, the workflow we present, which we call Lightweight Versioned Data (LVD), borrows best-practices for software
development and applies them to the challenge of maintaining and distributing
versioned data. Software developers maintain a core set of code which produces
a binary executable file that can be installed on a users local  computers. With a LVD system, data developers maintain a core set of files, which produce an
organised dataset that can be similarly be ``installed'' on a user's local
computer. In either the development of software or data, successive versions
are released over time (called ``releases''). The similarity in workflow
between sofwtare and data then allows us to deploy the same technological
platforms that are used in software development, for the development and
distribution of data. Importantly, LVD uses well-
established and tools, which ensures high-level performance and stability.

The core technologies used are summarised in Table \ref{tab:technologies} and
described further below.

\subsection{Version control via `git'}

\texttt{Git} is a version control system that tracks line-by-line changes in text
files. Git is used extensively in software development, but is increasingly
also being applied to scientific code and data management \citep{Ram-2013,
Perkel-2016}. Git is attractive for data management because it: 
\begin{itemize}
  \item Tracks all changes in monitored files. These changes are visible to
  anyone interacting with the repository.
  \item Allows users to annotate changes (`commits') with informative messages
  detailing the changes.
  \item Is a distributed system, meaning anyone who downloads the repository
  can access the full history and contribute changes.
  \item Can easily handle file sizes of at least 50Mb and repositories up to 1Gb.
\end{itemize}

\subsection{GitHub platform for storing git repositories}

The website GitHub (\smurl{www.github.com}) is a commercial platform for
hosting and interacting with git repositories. In practice, each dataset
should be a separate git repository on github (see Table \ref{tab:examples} for
examples). Although mainly used for
computer code, Github is now also being use to manage scientific data
\citep{Perkel-2016}. Github has a double role in our workflow:
\begin{enumerate}
  \item It provides a platform for data contributors to sync their files
  and correspond about changes in the dataset.
  \item It hosts a stream of data releases, thus acting as a central
  point for distribution of the data.
\end{enumerate}


Particular features of the Github platform that make it attractive for
handling versioned data include:

\begin{itemize}
  \item \textbf{Issues}: Each repository has tab where contributors can document issues in the data. These issue tags then provide a centralised point for describing a problem and discussing potential solutions.
  \item \textbf{Pull requests}: Much of the activity on GitHub is on open-source projects with many contributors. An essential feature that of sending a pull request, whereby a would-be contributor submits prospective changes to be incorporated.
  \item \textbf{Releases}: GitHub is designed to host releases (in this case compiled datasets), associated with a particular commit to the repository.
  \item \textbf{Tags}: Git commits and releases can be tagged with an informative label.
  \item \textbf{It's free}: All open-source projects are hosted for free on GitHub.
  \item \textbf{Quality}: The GitHub servers are designed for managing enetrpise level projects, and as such can deliver multiple requests for releases and have sophisticated capabilities.
  \item \textbf{DOIs}: Github repos can be automatically archived and assigned a DOI, via several providers (Table \ref{tab:doi_minting}).
\end{itemize}

\subsection{Semantic versioning}

To address the issue of how successive releases should be labelled, we suggest
applying the theory of ``semantic versioning'' developed for software
distribution. This system uses (the potentially familiar) notation of form
``1.0.0'' for successive versions (Fig \ref{fig:semantic}). As has been noted
(REFS), labelling of data versions presents a similar problem to the labelling
of software releases: signalling to users the types of change that has
occurred between successive versions. Applying semantic versioning to data, a
change from 1.0.0 to 1.0.1 implies tiny changes, for example a minor error
correction. A change from 1.0.0 to 1.1.0 implies a substantial enhancement,
for example adding a new study to a meta-analysis dataset (while otherwise[]
maintaining the same dataset structure). A change from 1.0.0 to 2.0.0 implies
a very major revision, for example improving the entire structure of the
dataset and adding new columns.

\subsection{\texttt{datastorr} and dataset-specific R packages}

While GitHub can host a stream of versioned data releases, many researchers will likely want to be able to access this data programmatically, and ideally, load it directly into their analysis platform. Our solution is for researchers releasing a dataset to write a customized R package, specific to their dataset. For example, the package
\texttt{baad.data} can retrieve data releases from the Biomass and Allometry Database, stored at \smurl{github.com/dfalster/baad/} (see Table \ref{tab:examples} for more examples). To enable this, we have developed a new R package called \texttt{datastorr}, which does all the heavy lifting both in terms of creating templates for the data-specific R packages and for interacting with the GitHub API. With \texttt{datastorr}, setting up a new data-specific package is straightforward and requires minimal computational skills. 

Specific features of the \texttt{datastorr} package that make it easy to use include:

\begin{itemize}
  \item \textbf{Speed}: The data packages contain only the code for retrieving the data, not the data itself, and so are quick to install.
  \item \textbf{Minimal code}: Specific versions of a dataset can be downlaoded in just two lines of code. Installing the package (if not already installed) takes an additional line.
  \item \textbf{Efficient}: On any given machine, each dataset is downloaded only once. It is then stored in a standard directory for R packages, and so is available for offline access.
\end{itemize}

[MWP: moved this from discussion]
To set up a
living dataset you need to provide:
\begin{enumerate}
  \item a github repository name (e.g.~`wcornwell/taxonlookup`, `traitecoevo/baad.data`).
  \item A single file that will be the versioned dataset (e.g. \texttt{plant\_lookup.csv} or \texttt{baad\_data.zip}).
  \item A function to load that data into R:(e.g.~a simple wrapper around \texttt{read.csv}).
\end{enumerate}

Then as the data changes (hopefully improves) with time, update the data
file and create a github release with a new version number (preferably
using \href{http://semver.org/}{semantic versioning}). All the versions
will be available to the user. And versions can be given a unique DOI
when appropriate, e.g.~when they are used in a publication.


We have created \texttt{datastorr} to make LVD easy to implement and manage for \texttt{R} users. And as we stated above, this has already been used to release a variety of datasets (Table \ref{tab:examples}). However, we can imagine alternative methods for interacting with GitHub and building templates; so while useful, adopting LVD does not critically depend on using \texttt{datastorr}. Furthermore, the concept of LVD is general and we would love to see implementations in \texttt{Python} or other languages. 

\section{Discussion}

Topics to consider:

\begin{itemize}

\item Addresses important need, leveraging open source platfroms. Shows power of open source: by writing single new package, can provide global distribution and access to data, with first class system for managing contents.
\item Versioning and open access essential for reproducibility
\item Versioning should enable better benchmarking of models, also because tests can be automated.
\item We have focussed on particualr set of tools, but could do similar things with other platfroms or languages.
....
\end{itemize}



%(old content unchanged)
%
%One of the key problems in switching from a static to a dynamic data
%world, is that it's currently hard to set up and maintain a system like
%this. Ideally using a versioned database should be easy for both the
%user and the data maintainer. This is especially crucial is the data
%maintainers are busy. Moreover, old versions need to be preserved in a
%stable and accessible way.
%
%To try and achieve to above goals, we built a system to distribute
%versions of our datasets. We think that with this infrastructure, it is
%relatively easy to set up and maintain a versioned system with any
%small-to-medium sized dataset.
%
%In our system, the data is downloaded only once per computer, available
%for offline access, easy to port to other platforms, and can get a doi
%when desired (i.e.~when it's used for a publication). All old versions
%are accessible.
%
%
%
%\subsection{How much expertise is needed to setup and use?}
%
%To use a versioned dataset: Very very basic \texttt{R}. With 4 lines of \texttt{R} code, you can
%have the data downloaded and loaded into R and ready for use.
%
%To set a system like this up: Basically there are two general skills required: 1) understanding
%version control via \texttt{git} and 2) writing a very simple R package.
%From the prolific Hadley Wickham there is a
%\href{http://r-pkgs.had.co.nz/git.html}{very clear introduction to git}
%and \href{http://r-pkgs.had.co.nz/intro.html}{an instruction to the
%basics of R packages}.


\subsection{Conclusion}

Because it builds on existing infra-structure for distributing software,
this system requires very little (or no) money to set up and only a
small amount of programming expertise. The current infrastructure will
not work for big or continually updated data (e.g.~weather stations or
satellite downloads). For a set of problems that need periodically
updated small-to-medium sized data (e.g.~most meta-analysis problems),
this might be a useful approach.

\newpage

\section{Tables}

\begin{table}[h!]
\centering
\caption{Alternative frameworks for setting up and managing a living database}
{\footnotesize
\vspace{1cm}
  \begin{tabular}{p{2.5cm}p{3.5cm}p{3.5cm}p{4cm}}
  \hline
  \textbf{Feature} & \textbf{Static datasets}& \textbf{Web database} & \textbf{Lightweight versioned data}\\
  \hline
   Webserver        & e.g. dataDryad & custom                          &  Maintained by github.com\\
   Backend          & none & SQL + Ruby-on-rails 			& git + datastorr \\
   User access      & Web browser & Web browser 				    & R \\
   Ease of setup    & Very easy & Hard 							& Easy\\ % server, backups, bandwidth
   Data size        & Up to several Gb & Small-Very large 				& Up to 1Gb\\
   Cost             & Varies & Varies  						& Free \\
   Bandwidth        &managed by dataDryad & Pro rata 						& Managed by github.com\\
   Maintainer skills &none & Ruby + php 					& R + git \\
   User skills      &Web browsing& Web browsing  					& basic R \\
   Versioning       &None& Hard 							& Easy \\
   DOI minting      &Automatic & Many steps 					& Automatic \\
  \hline 
  \\
 
  \end{tabular}
  } 
\label{tab:sql_v_versioneddata}
\end{table}

\newpage

\begin{table}[h!]
\centering
\caption{Technologies used to maintain, store and distribute "Lightweight versioned data".}
{\footnotesize
\vspace{1cm}
  \begin{tabular}{p{5cm}p{10cm}}
  \hline
  \textbf{Technology} & \textbf{Description} \\\hline
   git & Open source version control system used for tracking progressive changes in a set of text files, typically computer code but also data\\
   github.com & A commercial webplatform for sharing, visualising and managing `git' repositories. Includes issue tracking\\
   R     &  Open source statsitcial and data processing language \\
   CRAN  &  Open source repository of packages for the R language \\
   datastorr & R package used to fetch versioned releases from github  \\
   DOI & Digital object identifier, which refers a user to a single stable digital object \\ 
   DOI ``minting'' & creation of a DOI's \\
   \hline
  \end{tabular}
  }
\label{tab:technologies}
\end{table}

\newpage

\begin{table}[h!]
\centering
\caption{Groups of users and their requirements}
{\footnotesize
\vspace{1cm}
  \begin{tabular}{p{2cm}p{5cm}p{7cm}}
  \hline
  \textbf{Group} & \textbf{Goal} & \textbf{Requirements} \\ \hline
  Maintainer & Create and distribute versioned datasets & Low technical overhead \\
    & & Easy workflow for releasing new versions \\
    & & Long term preservation \\
    & & Easy to crowd-source error checking and contributions \\
    & & Low initial cost \\
    & & Low on-going maintenance \\
  Contributor & Contribute to future versions & Add new data \\
    & & Report errors in existing data  \\
  Users (all) & Easy access to releases & Introduction \& overview \\
    & & Long term stability \\
    & & Clear path for users to become contributors \\
  Users (computational, machines) & Build reproducible products on these data & Scripted access to releases\\
    & &Easy installation\\
    & & Long term stability \\
  \hline
  \end{tabular}
  }
\label{tab:user_requirements}
\end{table}

\newpage


\begin{table}[h!]
\centering
\caption{Alternative platforms for minting DOIs from github repositories}
{\footnotesize
\vspace{1cm}
  \begin{tabular}{p{4cm}p{8cm}}
  \hline
  \textbf{Provider} & \textbf{Features} \\ \hline
  Zenodo: \smurl{zenodo.org} & Automated DOI for each github release \\
    & Funded and maintained by CERN and OpenAIRE for long term preservation \\
    & Only includes files embedded in git repo, not built  by code\\
  Figshare: \smurl{figshare.com} & Builds data package based on github release \\
    & Allows you to upload additional files, e.g. datasets built by code \\
    & Integrated with several journals, e.g. Nature family, PNAS \\
    & Some limited potential for versioned DOIs\\
  \hline
  \end{tabular}
  }
\label{tab:doi_minting}
\end{table}

\newpage

\begin{table}[h!]
\centering
\caption{Example datasets using the lightweight versioned data workflow described in this paper.}
{\footnotesize
\vspace{1cm}

  \begin{tabular}{p{3.5cm}p{3cm}p{7cm}}
  \hline
   \textbf{Github repo} & \textbf{R package} & \textbf{Description} \\ \hline
  \ghsmurl{dfalster/baad/} & \texttt{baad.data} & The \texttt{Biomass And Allometry Database} \citep{Falster-2015} provides...\\
  \ghsmurl{traitecoevo/taxonlookup} & \texttt{taxonlookup} & \texttt{taxonlookup} \citep{Pennell-2015a} provides...\\
  \ghsmurl{github.com/...} & \texttt{plantgrowthform}  & \texttt{plantgrowthform} provides... \\
  \ghsmurl{github.com/...} & \texttt{fungaltraits} & \texttt{fungaltraits} (in progress) provides... \\
  \ghsmurl{github.com/...} & \texttt{13C} & \texttt{13C} (in progress) provides .... \\
  \hline
  \end{tabular}
  }
\label{tab:examples}
\end{table}

\newpage
\section{Figures}

\begin{figure}[!hb]
\centering
%\includegraphics[width=\linewidth]{figure.pdf}
\caption{Semantic versioning allows users to anticipate the types of changes that have occurred between successive versions of a dataset.
\textbf{a)} 3 numbers indicate type of change.
\textbf{b)} A typical release stream.}
\label{fig:semantic}
\end{figure}

\newpage


\begin{figure}[!hb]
\centering
%\includegraphics[width=\linewidth]{figure.pdf}
\caption{Overview of the technologies involved in distributing Lightweight versioned data.}
\label{fig:technology_stack}
\end{figure}
\bibliography{refs}

\end{document}

